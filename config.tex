% !TeX root = main.tex

\pagestyle{fancy}
\fancyhf{}
\fancyhead[LE,RO]{\leftmark}
\fancyfoot[CE,CO]{\thepage}

\textwidth=450pt
\oddsidemargin=0pt

\setcounter{tocdepth}{2}
\setcounter{secnumdepth}{3}

%\renewcommand{\cftpartleader}{\cftdotfill{\cftdotsep}} % for parts
%\renewcommand{\cftchapleader}{\cftdotfill{\cftdotsep}} % for chapters

\newcommand{\specialcell}[2][c]{%
  \begin{tabular}[#1]{@{}c@{}}#2\end{tabular}}

%\titleformat{\chapter}{\Huge\bfseries}{\chaptername\ \thechapter}{0pt}{\vskip 100pt\raggedright}%
% Alter <after-sep> in the macro below to vary the separation after the \chapter title.
%\titlespacing{\chapter}{0pt}{50pt}{50pt}
% \titlespacing{<command>}{<left>}{<before-sep>}{<after-sep>}[<right>]

% configura i lettori pdf
\hypersetup{%
  pdfpagemode={UseNone},
  hidelinks,                  % nasconde i collegamenti (non vengono quadrettati)
  hypertexnames=false,
  linktoc=all,                % inserisce i link nell'indice
  unicode=true,               % usa solo caratteri Latini nei segnalibri di Acrobat
  pdftoolbar=false,           % nasconde la toolbar di Acrobat
  pdfmenubar=false,           % nasconde il menu di Acrobat
  plainpages=false,
  breaklinks,
  pdfstartview={Fit},
  pdflang={it}
}

\lstdefinelanguage{Kotlin}{
  comment=[l]{//},
  commentstyle={\color{gray}\ttfamily},
  emph={filter, first, firstOrNull, forEach, lazy, map, mapNotNull, println},
  emphstyle={\color{OrangeRed}},
  identifierstyle=\color{black},
  keywords={!in, !is, abstract, actual, annotation, as, as?, break, by, catch, class, companion, const, constructor, continue, crossinline, data, delegate, do, dynamic, else, enum, expect, external, false, field, file, final, finally, for, fun, get, if, import, in, infix, init, inline, inner, interface, internal, is, lateinit, noinline, null, object, open, operator, out, override, package, param, private, property, protected, public, receiveris, reified, return, return@, sealed, set, setparam, super, suspend, tailrec, this, throw, true, try, typealias, typeof, val, var, vararg, when, where, while},
  keywordstyle={\color{NavyBlue}\bfseries},
  morecomment=[s]{/*}{*/},
  morestring=[b]",
  morestring=[s]{"""*}{*"""},
  ndkeywords={@Deprecated, @JvmField, @JvmName, @JvmOverloads, @JvmStatic, @JvmSynthetic, Array, Byte, Double, Float, Int, Integer, Iterable, Long, Runnable, Short, String, Any, Unit, Nothing},
  ndkeywordstyle={\color{BurntOrange}\bfseries},
  sensitive=true,
  stringstyle={\color{ForestGreen}\ttfamily},
}

\setminted[terraform]{
  linenos=true,
  breaklines=true,
  encoding=utf8,
  fontsize=\footnotesize,
  frame=lines
}

\setminted[java]{
  linenos=true,
  breaklines=true,
  encoding=utf8,
  fontsize=\footnotesize,
  frame=lines
}

\setminted[docker]{
  linenos=true,
  breaklines=true,
  encoding=utf8,
  fontsize=\footnotesize,
  frame=lines
}

\setminted[yaml]{
  linenos=true,
  breaklines=true,
  autogobble,
  encoding=utf8,
  fontsize=\footnotesize,
  frame=lines
}

\setminted[swift]{
  linenos=true,
  breaklines=true,
  autogobble,
  encoding=utf8,
  fontsize=\footnotesize,
  frame=lines
}

\setminted[sql]{
  linenos=true,
  breaklines=true,
  autogobble,
  encoding=utf8,
  fontsize=\footnotesize,
  frame=lines
}

\setminted[bash]{
  linenos=true,
  breaklines=true,
  autogobble,
  encoding=utf8,
  fontsize=\footnotesize,
  frame=lines
}

\setminted[toml]{
  linenos=true,
  breaklines=true,
  autogobble,
  encoding=utf8,
  fontsize=\footnotesize,
  frame=lines
}

\setminted[kotlin]{
  linenos=true,
  breaklines=true,
  autogobble,
  encoding=utf8,
  fontsize=\footnotesize,
  frame=lines
}

\setminted[ruby]{
  linenos=true,
  breaklines=true,
  encoding=utf8,
  fontsize=\footnotesize,
  frame=lines
}

\setminted[json]{
  linenos=true,
  breaklines=true,
  encoding=utf8,
  fontsize=\footnotesize,
  frame=lines
}