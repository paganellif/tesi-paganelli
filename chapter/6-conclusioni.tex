% !TeX root = ../main.tex

% dire i risultati ottenuti come ad esempio tempistiche delle pipeline per i vari casi (con cache locale, con cache shared, modifica solo al codice shared, solo alla app android o tutto, ...), cosa non è stato fatto

\begin{table}[H]
\centering
    \begin{tabular}{|c|c|c|c|}
         \hline
         \textbf{Piattaforma} & \textbf{Rilascio Alpha} & \textbf{Rilascio Beta} & \textbf{Analisi}\\
         \hline
         Android & $\sim$ \textit{20 min} & $\sim$ \textit{1 min} & $\sim$ \textit{15 min} \\
         \hline
         iOS & \textit{?} & \textit{?} & $\sim$ \textit{15 min} \\
         \hline
         Android e iOS & \textit{?} & \textit{?} & \textit{?} \\
         \hline
    \end{tabular}
    \caption{Tempistiche di alcuni dei principali task del processo automatico, ottenute tramite la piattaforma GitLab}
\end{table}

\section{Sviluppi futuri}
Dati i risultati ottenuti sia nella progettazione del processo di sviluppo automatizzato per le applicazioni mobile (capitolo \ref{ch:ch4}) che nello sviluppo della applicazione PoC MaggioliEbook (capitolo \ref{ch:ch5}) i seguenti task rappresentano possibili sviluppi futuri in merito al lavoro svolto in questo progetto di tesi:

\begin{itemize}
    \item \textit{Mobile Application Development Lifecycle}
    \begin{itemize}
        \item Studio e ricerca approfondita per la fase di monitoring, sia della applicazione che dell'utente.
        \item Utilizzo del meccanismo \textit{Remote Configuration}\footnote{\url{https://firebase.google.com/docs/remote-config}} per modificare aspetti della applicazione in modo dinamico senza dover rilasciare nuove versioni. Un esempio tipico è la migrazione dei database: grazie al meccanismo di remote configuration è possibile settare da remoto il \textit{jdbc}\footnote{Java DataBase Connectivity} url che permette di connettersi al database senza dover rilasciare una nuova versione della applicazione con il valore cablato nel codice.
        \item Valutazione di alternative all'hardware Apple fisico. Alcune possibilità individuate sono: ($i$) runner gestiti con sistema operativo MacOS (disponibili su diverse piattaforme come GitLab\footnote{\url{https://docs.gitlab.com/ee/ci/runners/saas/macos_saas_runner.html}}, GitHub\footnote{\url{https://docs.github.com/en/actions/using-github-hosted-runners/about-github-hosted-runners}} e CircleCI\footnote{\url{https://circleci.com/docs/using-macos}}), ($ii$) virtual machine as-a-Service con sistema operativo MacOS (disponibili tra i servizi cloud Amazon\footnote{\url{https://aws.amazon.com/ec2/instance-types/mac/}}) e ($iii$) immagine Docker MacOS\footnote{\url{https://github.com/sickcodes/Docker-OSX}}.
        \item Valutazione soluzioni complete as-a-Service. Ad esempio \textit{Bitrise}\footnote{\url{https://www.bitrise.io/}} o XCode Cloud\footnote{\url{https://developer.apple.com/documentation/xcode/xcode-cloud}} (solamente per la parte Apple).
        \item Sperimentazione tecniche CI/CD con tecnologie cross-platform come \textit{Ionic}\footnote{\url{https://ionicframework.com/}}, \textit{Flutter}\footnote{\url{https://flutter.dev/}} o \textit{React Native}\footnote{\url{https://reactnative.dev/}} e stesso caso d'uso (applicazione PoC MaggioliEbook) in modo da avere dei parametri di riferimento e confronto.
    \end{itemize}
    \item \textit{Applicazione PoC MaggioliEbook}
    \begin{itemize}
        \item Implementazione della visualizzazione dei documenti "statici", ovvero in formato PDF con la relativa gestione di annotazioni e segnalibri.
        \item Gestione dei contenuti aggiuntivi dei documenti digitali in formato EPUB oltre alle annotazioni e ai segnalibri come ad esempio pagine interattive (quiz) e visualizzazione norme legislative provenienti da fonti esterne.
        \item Gestione annotazioni e segnalibri remota invece che locale sul dispositivo dell'utente.
    \end{itemize}
\end{itemize}
