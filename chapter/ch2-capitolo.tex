% !TeX root = ../main.tex

\section{Analisi dei Requisiti}
\subsection{Requisiti Funzionali}
\begin{itemize}
    \item \textbf{R1} - Visualizzare documenti.
    \begin{itemize}
        \item \textbf{R1.1} - In modo fluido, mostrando il contenuto adattato al dispositivo in cui viene mostrato.
        \item \textbf{R1.2} - In modo statico, mostrando il contenuto con uno specifico layout indipendente dal dispositivo in cui viene mostrato.
    \end{itemize}
    \item \textbf{R2} - Modificare documenti in modo fluido (lato utente).
    \begin{itemize}
        \item \textbf{R2.1} - Aggiungere commenti, sottolineature, evidenziazioni, al contenuto fluido.
        \item \textbf{R2.2} - Memorizzare commenti, sottolineature, evidenziazioni apportate al contenuto fluido.
    \end{itemize}
    \item \textbf{R3} - Gestione utente.
    \begin{itemize}
        \item \textbf{R3.1} - Login (autenticazione) utente.
        \item \textbf{R3.2} - Visualizzare documenti a cui l'utente è abbonato.
    \end{itemize}
    \item \textbf{R4} - Ricerca documenti
    \begin{itemize}
        \item \textbf{R4.1} - Ricerca documenti senza autenticazione. Qualsiasi utente può effettuare una ricerca dei documenti disponibili, senza ottenere il contenuto.
        \item \textbf{R4.2} - Ricerca avanzata in base a diversi campi. %TODO: da definire quali campi
    \end{itemize}
    \item \textbf{R5} - Convertire documenti da modo statico a modo fluido.
    \item \textbf{R6} - Modificare documenti in modo fluido (lato azienda).
    \begin{itemize}
        \item \textbf{R6.1} - Aggiungere elementi/contenuti al documento in modo fluido (hyperlink, quiz, video, immagini, ...). %TODO: definire quali cose sono da aggiungere al documento
        \item \textbf{R6.2} - Memorizzare gli elementi/contenuti apportati al documento in modo fluido (hyperlink, quiz, video, immagini, ...).
    \end{itemize}
\end{itemize}

\subsection{Requisiti Non Funzionali/Tecnologici}
\begin{itemize}
    \item \textbf{T1} - Applicazione nativa Android e iOS, sfruttando Kotlin Multiplatform Mobile (KMM).
    \item \textbf{T2} - Continuous Integration e Continuous Delivery
    \begin{itemize}
        \item \textbf{T2.1} - Analisi statica del codice (SAST\footnote{Static Application Security Testing}).
        \item \textbf{T2.2} - Unit testing, code coverage e E2E\footnote{end-to-end} testing.
        \item \textbf{T2.3} - Rilascio automatico nei relativi store delle piattaforme scelte (Google Play per Android e App Store per iOS).
    \end{itemize}
    \item \textbf{T3} - Monitoraggio applicazione (Analytics).
\end{itemize}

\section{Analisi Formati Digitali Fluidi}
I documenti attualmente sono reperibili in formato PDF e/o HTML. Il formato PDF è quello con cui i documenti vengono effettivamente archiviati: per ottenere un documento in formato HTML è necessario utilizzare un servizio interno, chiamato \textit{pdf2html}, il quale effettua la conversione. Entrambi i formati rispettano i requisiti per i documenti definiti "statici" ma non per quelli definiti "fluidi":
\begin{itemize}
    \item \textbf{PDF} (Portable Document Format) - Formato di file sviluppato da Adobe per rappresentare documenti di testo e immagini in modo indipendente dall'hardware e dal software utilizzati per generarli o per visualizzarli. Viene dunque generato e visualizzato con uno specifico layout.
    \item \textbf{HTML} (HyperText Markup Language) - Linguaggio di formattazione che descrive le modalità di impaginazione o visualizzazione grafica (layout) del contenuto, testuale e non, di una pagina web attraverso tag di formattazione. Viene generato tramite conversione del documento PDF riportando fedelmente il layout iniziale.
\end{itemize}
Per soddisfare i requisiti \textit{R1.1}, \textit{R2.1}, \textit{R5} e \textit{R6.1} il formato "fluido" deve:
\begin{itemize}
    \item rappresentare solamente il contenuto dei documenti "statici", rimuovendo tutte le formattazioni di layout,
    \item essere modificabile,
    \item poter essere ricavato convertendo un documento attualmente in formato "statico" (ovvero deve esistere un algoritmo/software per poter effettuare la conversione).
\end{itemize}
I formati attualmente disponibili che soddisfano i requisiti sopra indicati rappresentano implementazioni dello standard Open eBook (OeB), elaborato dall'Open E-book Forum. Tra questi i formati più diffusi sono:
\begin{itemize}
    \item \textbf{MOBI} (Mobipocket) - Standard proprietario (\textit{Amazon}) per la pubblicazione di libri digitali (eBook).\\
    Principali caratteristiche:
    \begin{itemize}
        \item basato sulla Open eBook standard utilizzando XHTML,
        \item annotazioni (highlights, segnalibri, correzioni, note e disegni) possono essere applicati, organizzati, e richiamati,
        \item può includere anche JavaScript e cornici.
    \end{itemize}
    \item \textbf{EPUB} (Electronic Publication) - Standard aperto specifico per la pubblicazione di libri digitali (eBook).\\
    Principali caratteristiche:
    \begin{itemize}
        \item basato sulla Open eBook standard utilizzando XML,
        \item a partire da settembre 2007 è lo standard ufficiale dell'International Digital Publishing Forum (IDPF)\footnote{\url{https://web.archive.org/web/20080827131750/http://www.idpf.org/2007/ops/OPS_2.0_final_spec.html}},
        \item CSS per il layout e la formattazione,
        \item testo "re-flowable" con grafica raster e vettoriale,
        \item disponibilità di diversi software, sia proprietari che open source, per la manipolazione del file (\textit{Adobe InDesign}, \textit{Sigil}, \textit{Calibre}, ...),
        \item disponibilità di tante librerie in diversi linguaggi per la manipolazione del file.
    \end{itemize}
\end{itemize}
Le caratteristiche determinanti che hanno portato alla scelta del formato EPUB sono state (\textit{i}) lo standard aperto e (\textit{ii}) la disponibilità, sia di software che di librerie, per la manipolazione del file. 

\subsection{Conversione PDF/HTML2EPUB}
La grande quantità di documenti già archiviati in formato PDF (> 2000) rende impossibile l'adozione di una procedura manuale di conversione e pertanto deve essere progettata in modo tale che questa venga effettuata on-demand dagli utenti finali tramite un apposito servizio REST.\\
Dati i formati in input (PDF o HTML) e stabilito il formato di output (EPUB) è necessario definire i tool e la procedura di conversione. Prima di analizzare eventuali librerie utili a implementare la procedura di conversione è stata effettuata una ricerca di tool già disponibili sul mercato in grado di svolgere il compito richiesto. I principali requisiti per la ricerca del tool sono:
\begin{itemize}
    \item possibilità di convertire da almeno uno dei due formati disponibili in input verso il formato di output,
    \item eseguibile binario senza GUI.
\end{itemize}

\subsubsection{Calibre}
Software libero, open source\footnote{\url{https://github.com/kovidgoyal/calibre}} e multipiattaforma, fornisce una suite di tool creare, conservare, catalogare e manipolare eBook. Tra i formati supportati, sia in input che output, è possibile trovare Epub, Fb2, Html, Lit, Mobi, Odt e Pdf.\\
Tra i vari tool della suite è presente un convertitore, chiamato \textit{ebook-convert}\footnote{\url{https://manual.calibre-ebook.com/generated/en/ebook-convert.html}}, il quale soddisfa i requisiti sopra indicati.

\begin{listing}[H]
\begin{minted}{bash}
brew install calibre
ebook-convert test.pdf test.epub
\end{minted}
\caption{Esempio semplice di conversione PDF/EPUB tramite \textit{ebook-convert}}
\end{listing}

\section{Pdf2Epub}