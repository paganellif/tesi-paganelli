% !TeX root = ../main.tex

\section{Introduzione}

\section{Riuso}
% dimostrazione del fatto che ho effettivamente incluso e utilizzato i template della cicd (che stanno in un altro repo) nel repo del progetto della app

\section{Stabilizzazione e rilascio}
% efficacia del rilascio automatico delle versioni android/ios su google play e testflight

\section{Analisi del codice}
% efficacia della pipeline di analisi

\section{Statistiche}
% qualche misurazione delle tempistiche di esecuzione delle pipeline nei vari casi (android, ios, ios+android e rilascio, analisi, ...)
% non ho parametri per fare il confronto rispetto a prima, la app sviluppata è nuova e altri team non hanno cicd app mobile

\section{Lavori futuri}
% cosa ho intenzione di fare dopo, continuazione del lavoro in azienda
\begin{itemize}
    \item \textit{SDLC e CI/CD}
    \begin{itemize}
        \item Studio, ricerca e sperimentazione per la fase di monitoring, sia della applicazione che dell'utente.
        \item Utilizzo del meccanismo \textit{Remote Configuration}\footnote{\url{https://firebase.google.com/docs/remote-config}} per modificare aspetti della applicazione in modo dinamico senza dover rilasciare nuove versioni. Un esempio tipico è la migrazione dei database: grazie al meccanismo di remote configuration è possibile settare da remoto il \textit{jdbc}\footnote{Java DataBase Connectivity} url che permette di connettersi al database senza dover rilasciare una nuova versione della applicazione con il valore cablato nel codice.
        \item Valutazione di alternative all'hardware Apple fisico. Come indicato nel capitolo \ref{ch:cicd} alcune possibilità individuate sono: (\textit{i}) runner gestiti con sistema operativo MacOS (disponibili su diverse piattaforme come GitLab\footnote{\url{https://docs.gitlab.com/ee/ci/runners/saas/macos_saas_runner.html}}, GitHub\footnote{\url{https://docs.github.com/en/actions/using-github-hosted-runners/about-github-hosted-runners}} e CircleCI\footnote{\url{https://circleci.com/docs/using-macos}}), (\textit{ii}) virtual machine as-a-Service con sistema operativo MacOS (disponibili tra i servizi cloud Amazon\footnote{\url{https://aws.amazon.com/ec2/instance-types/mac/}}) e (\textit{iii}) immagine Docker MacOS\footnote{\url{https://github.com/sickcodes/Docker-OSX}}.
        \item Valutazione soluzioni complete as-a-Service. Ad esempio \textit{Bitrise}\footnote{\url{https://www.bitrise.io/}} o XCode Cloud\footnote{\url{https://developer.apple.com/documentation/xcode/xcode-cloud}} (solamente per la parte Apple).
        \item Sperimentazione tecniche CI/CD con tecnologie cross-platform come \textit{Ionic}\footnote{\url{https://ionicframework.com/}}, \textit{Flutter}\footnote{\url{https://flutter.dev/}} o \textit{React Native}\footnote{\url{https://reactnative.dev/}} e stesso caso d'uso (applicazione PoC MaggioliEbook) in modo da avere dei parametri di riferimento e confronto.
    \end{itemize}
    \item \textit{Applicazione MaggioliEbook}
    \begin{itemize}
        \item Implementazione della visualizzazione dei documenti "statici" (PDF) con la relativa gestione di annotazioni e segnalibri.
        \item Gestione dei contenuti aggiuntivi dei documenti digitali in formato EPUB oltre alle annotazioni e ai segnalibri come ad esempio pagine interattive (quiz) e visualizzazione norme legislative provenienti da fonti esterne.
        \item Gestione annotazioni e segnalibri remota invece che locale sul dispositivo dell'utente.
    \end{itemize}
\end{itemize}