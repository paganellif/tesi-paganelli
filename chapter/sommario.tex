% !TeX root = ../main.tex

Distribuire un'applicazione che soddisfi i requisiti del cliente e che sia in grado di accogliere rapidamente eventuali modifiche è, 
al giorno d’oggi, 
d'obbligo per aziende che si occupano di applicazioni mobile che vogliano rimanere competitive sul mercato.
Il principale fattore chiave in grado di mantenere una azienda al passo con la continua evoluzione sia del mercato che delle tecnologie mobile è la continua innovazione e ottimizzazione dell'intero processo di sviluppo.

La collaborazione e la comunicazione tra diversi team, 
l'utilizzo di cicli iterativi di sviluppo, 
i rilasci frequenti e l'automazione dei test sono alcune delle pratiche incentivate dalla cultura DevOps, 
adottabile con successo (si vedrà in questa tesi) anche per lo sviluppo di applicazioni mobile.

Innovare e ottimizzare il processo di sviluppo non significa solo automatizzare l'esecuzione dei task.
Un ruolo importante è giocato anche da aspetti legati all'applicazione:
l'architettura,
il paradigma di programmazione,
e gli strumenti utilizzati.
In particolare,
al fine di applicare il principio ``\textit{Don't repeat yourself}'' (DRY) e semplificare la manutenzione,
diversi moderni framework per lo sviluppo di applicazioni mobile,
detti ``multipiattaforma'', 
propongono meccanismi che consentono di condividere codice tra piattaforme differenti.

L'obiettivo di questa tesi è dunque quello di discutere
(capitoli \ref{ch:devops}, \ref{ch:sdlc} e \ref{ch:app-multiplatform})
e mostrare, applicate ad un caso di studio industriale
(capitoli \ref{ch:casodistudio}, \ref{ch:cicd} e \ref{ch:maggioliebook}),
l'uso di tecniche DevOps
nell'ambito di applicazioni mobile,
ed in particolare mostrando come queste siano applicabili in congiunzione
ai framework di sviluppo multipiattaforma (in particolare, Kotlin Multiplatform).
