% !TeX root = ../main.tex

Distribuire la giusta applicazione, con le giuste funzionalità, al giusto cliente e nel giusto tempo è al giorno d'oggi un obbligo per una azienda che sviluppa applicazioni mobile e che vuole rimanere competitiva sul mercato. Il principale fattore chiave in grado di mantenere una azienda al passo con la continua evoluzione sia del mercato che delle tecnologie mobile è la continua innovazione e ottimizzazione dell'intero processo di sviluppo delle applicazioni mobile.

La collaborazione e la comunicazione tra diversi team, l'utilizzo di cicli iterativi di sviluppo, i rilasci frequenti e l'automazione dei test sono soltanto alcuni esempi di pratiche incentivate dalla cultura DevOps, la quale può essere adottata efficacemente anche per lo sviluppo di applicazioni mobile come dimostrato in questa tesi.

Innovare e ottimizzare il processo di sviluppo però non significa solo automatizzare l'esecuzione dei task. Un ruolo importante è giocato infatti anche dagli aspetti legati direttamente alla applicazione che si sviluppa come ad esempio l'architettura, il paradigma di programmazione e gli strumenti utilizzati. Tutti questi aspetti vengono trattati dai moderni framework per lo sviluppo di applicazioni mobile basati sulla condivisione del codice tra diverse piattaforme target completamente differenti tra loro.

L'obiettivo di questa tesi è dunque quello di definire in prima battuta la cultura DevOps, il ciclo di sviluppo delle applicazioni mobile e i framework di sviluppo multipiattaforma che lo supportano (capitoli \ref{ch:devops}, \ref{ch:sdlc} e \ref{ch:app-multiplatform}) al fine di mostrare successivamente la loro efficacia nella realizzazione di un caso di studio industriale (capitoli \ref{ch:casodistudio}, \ref{ch:cicd} e \ref{ch:maggioliebook}). Sono infine descritti i risultati ottenuti (capitoli \ref{ch:risultati} e \ref{ch:conclusioni}) dallo sviluppo di una applicazione multipiattaforma sfruttando il processo progettato e il sistema di automazione realizzato.
