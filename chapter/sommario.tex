% !TeX root = ../main.tex

\parindent=0pt
Nonostante la presenza di diversi team dedicati allo sviluppo di applicazioni mobile all'interno del Gruppo Maggioli, non esiste un chiaro processo standard a livello aziendale per lo sviluppo di applicazioni mobile. La situazione è invece differente per la maggior parte degli altri team che si occupano dello sviluppo di altre tipologie di applicazioni, come ad esempio Web Application, per i quali invece esiste un processo ben definito e collaudato a livello aziendale.\\
Questo progetto di tesi magistrale consiste ($i$) nella progettazione di un primo modello di standard aziendale per il processo di sviluppo delle applicazioni mobile tale da poter essere adottato dagli altri team in Maggioli, ($ii$) nella applicazione delle pratiche di automazione DevOps a tale processo e ($iii$) nella dimostrazione della sua efficacia tramite lo sviluppo di un PoC con lo scopo di aumentare la conoscenza e l'esperienza in ambito applicazioni mobile del team di Ricerca e Sviluppo in Maggioli.\\
Il primo task svolto, descritto nel capitolo \ref{ch:ch2}, ha l'obiettivo di definire i vari passi che compongono il tipico processo di sviluppo di applicazioni mobile basandosi sia sulle tecniche attualmente adottate dai team in Maggioli che dalle best practice individuate in letteratura. Successivamente sono stati scelti gli strumenti necessari per soddisfare i requisiti dettati dal processo di sviluppo: nel capitolo \ref{ch:ch3} sono descritti gli strumenti core utilizzati nel processo di sviluppo (\textit{GitLab} e \textit{Fastlane}) e nello sviluppo della applicazione PoC (\textit{Kotlin Multiplatform Mobile}). Dato il processo e i tool, viene descritto nel capitolo \ref{ch:ch4} il processo progettato tramite l'applicazione delle pratiche DevOps come \textit{Continuous Integration}, \textit{Continuous Delivery} e \textit{Continuous Inspection}: l'obiettivo è quello di integrare in modo più frequente piccole modifiche (integration) riducendo il tempo che intercorre tra la modifica del codice e l'effettivo rilascio (delivery) della applicazione sullo store della relativa piattaforma target, con un aumento della qualità e della sicurezza (inspection). Realizzato il sistema a supporto del processo di sviluppo, quest'ultimo è stato testato tramite lo sviluppo di una applicazione PoC, chiamata \textit{MaggioliEbook}, descritta nel capitolo \ref{ch:ch5}. L'applicazione rappresenta un primo passo per l'azienda verso una nuova modalità di fruizione e consultazione dei propri contenuti editoriali digitali in formato EPUB e su dispostivi mobile. Infine nel capitolo \ref{ch:ch6} sono analizzati i risultati ottenuti da questo progetto di tesi e indicati possibili lavori futuri che verranno svolti.
