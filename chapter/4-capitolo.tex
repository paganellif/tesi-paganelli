% !TeX root = ../main.tex

Dato il processo di sviluppo delle applicazioni mobile individuato nel capitolo \ref{ch:ch3} e definiti gli strumenti e task necessari, si descrive in questo capitolo come è stato automatizzato il processo adottando le moderne tecniche di integrazione continua e rilascio continuo al fine di fornire un modello di processo di sviluppo ai vari reparti aziendali che si occupano di applicazioni mobile.

\section{Continuous Integration}

\subsection{Build}

\subsection{Testing}

\subsection{Analysis}
In questa sezione si descrivono le tecniche di analisi statica del codice adottate e come queste sono state integrate nel processo di sviluppo per controllare qualità del codice, code smell, bug e vulnerabilità.
% sonarqube (android,shared,ios), detekt (android,shared), dependency-check (android,shared,ios)
% defectdojo per raccogliere in un unico punto centralizzato tutti i report prodotti

\subsection{Package}

\section{Continuous Delivery}

\subsection{Alpha Release}

\subsection{Beta Release}

\subsection{Production Release}

\section{Continuous Monitoring}
\subsection{Monitoring}
\subsection{Analytics}

\section{Infrastruttura}
% descrivere l'infrastruttura necessaria a supporto della cicd runners, nexus, defectdojo, ....

\section{Templating}
% definire la cicd in template in un progetto a parte in modo che possano essere importati nel PoC (ed essere usati da altri in futuro)
% dire come gitlab permette di farlo, quali sono i meccanismi ecc ecc
% lo stesso risultato può essere ottenuto distribuendo delle github action
