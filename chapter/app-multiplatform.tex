% !TeX root = ../main.tex

\section{Introduzione}
La continua crescita della quantità di dispositivi mobile presenti, e in particolare gli smartphone, ha reso molto importante a livello globale il mercato delle applicazioni e per questo motivo sono sempre di più le aziende che decidono di investire risorse nello sviluppo e nella vendita di applicazioni mobile. Una azienda che intende targettizzare il maggior numero di possibili utenti con le proprie applicazioni deve considerare che l'intero mercato è spartito in base alla diffusione dei sistemi operativi per smartphone. Ad oggi, i sistemi operativi più diffusi sono Android (Google) e iOS (Apple), i quali coprono quasi la totalità del mercato con quote rispettivamente del 71\% e del 28\%\footnote{\href{https://www.statista.com/statistics/272698/global-market-share-held-by-mobile-operating-systems-since-2009/}{https://www.statista.com/statistics/272698}}. Questi dati comportano la necessità di sviluppare la stessa applicazione per almeno due piattaforme completamente differenti.

% cosa si intende con applicazione multipiattaforma, differenza rispetto alle app normali e differenze tra multiplatform e crossplatform.

\section{Vantaggi}
% vantaggi derivanti dallo sviluppo multi/cross platform (risparmio risorse, riuso, performance, sdk nativo, ...)

\section{Strumenti}
% tool esistenti individuati (KMM, ionic, flutter, react native)
% requisiti comuni dei vari tools (ad esempio avere mac apple per compilazione app ios: in ogni caso prima o poi deve essere compilata una app nativa ipa per ios)

\section{Kotlin Multiplatform Mobile}
% descrizione dettagliata di KMM e il suo funzionamento