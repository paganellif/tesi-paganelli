% !TeX root = ../main.tex

\section{Introduzione}
% nel capitolo precedente è stato introdotto il caso di studio industriale e definita la pipeline obiettivo che si vuole realizzare per automatizzare il processo di sviluppo. in questo capitolo viene descritto cosa è stato effettivamente fatto per raggiungere l'obiettivo.

\section{Modello di branching}
% workflow aziendale adattato per lo sviluppo app mobile
L’utilizzo di un adeguato flusso di lavoro è fondamentale per definire una efficiente automazione CI/CD. Con branching si intende l’utilizzo di uno o più flussi principali dai quali divergono altri flussi per svolgere determinati lavori per poi convergere al loro termine: in base alle modalità di apertura e chiusura di questi flussi si definiscono diversi modelli di branching.

\section{Continuous Integration}
% progettazione e implementazione CI
\subsection{Build e Packaging}

\subsection{Testing}

\subsection{Dependency Management}

\section{Continuous Delivery}
% progettazione e implementazione CD

\subsection{Alpha/Beta Release}

\section{Continuous Inspection}
% progettazione e implementazione analisi

\section{Templating}
% strumenti utilizzati per favorire il riuso della cicd realizzata

\section{Self-Hosted MacOS GitLab Runner}
% runner macos: in azienda non esistono runner macos.. riprendere il problema del fatto che apple obbliga a usare macOS, indicare le possibili soluzioni (runner managed/self-hosted, ecc) e come ho configurato il runner self-hosted
% approfondire tipologie di runner executor in gitlab e perche ho usato l'executor shell