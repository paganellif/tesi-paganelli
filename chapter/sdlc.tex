% !TeX root = ../main.tex

\section{Introduzione}
In questo capitolo viene analizzato il processo di sviluppo tipico delle applicazioni mobile al fine di porre le basi per la progettazione dello stesso processo di sviluppo,
applicando le pratiche e le tecniche DevOps d'automazione viste nel capitolo precedente. 
Lo scopo di questa fase iniziale di progetto è quindi la definizione di tutti i principali task e sotto-task necessari allo sviluppo di applicazioni mobile, 
dalla scrittura del codice sorgente al rilascio sui marketplace delle relative piattaforme target. 

A prescindere dalla tipologia di applicazione, il processo di sviluppo è composto concettualmente dalla stessa sequenza di fasi di lavoro: 
(\textit{i}) pianificazione, 
(\textit{ii}) progettazione, 
(\textit{iii}) sviluppo, 
(\textit{iv}) stabilizzazione, 
(\textit{v}) rilascio e (\textit{vi}) monitoraggio. 
Ognuna di queste macro-fasi comprende a sua volta una serie di sotto-fasi tra le quali esistono specifiche dipendenze:

\begin{figure}[H]
    \centering
    \includegraphics[width=0.76\textwidth]{img/sdlc.png}
    \caption{Ciclo di vita di sviluppo tipico delle applicazioni mobile}
    \label{sdlc-app-mobile-fig}
\end{figure}

Il processo di sviluppo delle applicazioni mobile è simile a quello di qualsiasi altra applicazione software ma in questo caso l'obiettivo è distribuire l'applicazione al fine di dare la possibilità all'utente finale d'installarla sul proprio dispositivo. 
Nei processi di sviluppo di altre tipologie di applicazioni, 
come per esempio le Web Application, 
l'obiettivo è invece quello di mettere in esecuzione l'applicazione in un ambiente target accessibile dall'utente finale.

\section{Pianificazione}
In questa prima fase del processo di sviluppo si formalizzano i requisiti, 
funzionali e non funzionali, 
che devono essere soddisfatti dall'applicazione per ottenere l'approvazione del committente e si pianifica il lavoro per le successive fasi in termini di task, risorse e tempo.

Tramite la definizione di casi d'uso si rappresentano le interazioni tra gli attori all'interno del sistema e i loro ruoli. 
Questi casi d'uso definiscono dei possibili scenari dove il sistema riceve delle richieste esterne, 
ad esempio l'input dell'utente,
e come esso risponde a quell'input. 
Dopo aver acquisito un numero appropriato di casi d'uso e attori è molto più semplice iniziare a progettare un'applicazione poiché è possibile concentrarsi su come creare l'applicazione anziché sulla definizione delle sue funzionalità.

\section{Progettazione}
La fase di progettazione dell'applicazione è composta da un insieme di sotto-task tra cui i principali sono (\textit{i}) la modellazione del dominio applicativo,
(\textit{ii}) la scelta dell'architettura da utilizzare e (\textit{iii}) la prototipazione dell'esperienza utente e dell'interfaccia grafica (UX/UI\footnote{User Experience/User Interface}).

Tipicamente la progettazione UX/UI viene svolta tramite l'ausilio di mockup per definire prima come l'utente intende utilizzare l'applicazione (esperienza) e poi aspetti grafici come colori,
font e icone (interfaccia).

\section{Sviluppo}
Una volta progettata l'applicazione è possibile partire con la fase di sviluppo. 
Solitamente l'obiettivo è far iniziare la fase di sviluppo il prima possibile in modo da sviluppare un prototipo funzionante, 
spesso chiamato \textit{Minimum Viable Product}~\cite{ries2011startup}, 
e ottenere la validazione da parte del committente, 
la quale è l'obiettivo principale della fase successiva di stabilizzazione.

\section{Stabilizzazione}
La stabilizzazione consiste nella risoluzione di problemi sia a livello funzionale che a livello d'usabilità e di prestazioni al fine d'ottenere una versione di applicazione pronta da distribuire. 

Adottare la cultura DevOps e la metodologia Agile significa applicare il concetto ``\textit{Release Early, Release Often}'': le funzionalità preziose devono essere raggruppate insieme e rilasciate prima per ottenere miglioramenti in termini di valore~\cite{shore2008art}. 
Questa parte del ciclo di sviluppo dovrebbe iniziare il più presto possibile in modo da individuare e risolvere i problemi prima che diventino un costo. 
Tipicamente per qualsiasi applicazione, 
anche non specifica per i dispositivi mobile, 
sono previste le seguenti sotto-fasi del processo di stabilizzazione\footnote{\href{https://docs.microsoft.com/itit/xamarin/cross-platform/get-started/introduction-to-mobile-sdlc}{https://docs.microsoft.com/itit/xamarin/cross-platform/get-started/introduction-to-mobile-sdlc}}:

\begin{itemize}
    \item \textbf{Prototype} - L'applicazione include soltanto alcune delle funzionalità principali e sono presenti bug maggiori. In questa fase il focus è sulla singola funzionalità implementata fornita dal prototipo per il testing.
    
    \item \textbf{Alpha} - Tutte le principali funzionalità sono completate e devono essere testate.
    
    \item \textbf{Beta} - Gran parte delle funzionalità, sia principali che ausiliarie, sono state completate e i bug maggiori sono stati risolti.
    
    \item \textbf{Release Candidate} - Tutte le funzionalità sono state completate e testate, ma potrebbero essere presenti ancora bug minori.
\end{itemize}

\subsection{Alpha}
La prima versione funzionante di un'applicazione è detta \textit{alpha} ed è utilizzata per il testing interno di specifiche funzionalità.
Questo significa che può presentare anche bug o funzionalità mancanti ma deve contenere almeno le funzionalità che devono essere testate per quella specifica versione.

Solitamente prima della release di un'applicazione, 
anche se in fase di testing, 
è necessario attendere la sua approvazione da parte del gestore del servizio: tale processo d'approvazione pre-release è detto \textit{App Review}. 
Nel caso del testing interno è possibile distribuire l'applicazione ad un insieme ristretto di tester. 
Continuando con i rilasci di versioni alpha vengono aggiunte nuove funzionalità e/o risolti eventuali bug: 
quando la versione è considerata pronta viene eseguita la sua promozione, 
ovvero il rilascio di una versione alpha in versione beta.

\subsection{Beta}
A questo punto del processo di sviluppo l'applicazione è considerata completa a tutti gli effetti,
a meno di bug e/o problemi di stabilità. 
La versione \textit{beta} rappresenta dunque la prima versione dell'applicazione resa disponibile ai tester esterni, 
ovvero quegli utenti che non hanno partecipato alle fasi di sviluppo e che svolgono il ruolo di validazione delle funzionalità. 
Si distinguono due tipologie di beta testing:

\begin{itemize}
    \item \textbf{Aperto} - L'applicazione è rilasciata per la fase di testing esterno permettendo l'accesso a qualsiasi utente con account da beta tester.
    
    \item \textbf{Chiuso} - L'accesso all'applicazione di test è limitato ad un insieme ristretto di tester, tipicamente gestiti tramite mailing list o link di condivisione.
\end{itemize}

Dopo aver ottenuto la validazione da parte dei tester, 
la quale potrebbe richiedere più iterazioni di sviluppo e rilascio di versioni alpha-beta,
anche per la versione beta si effettua la promozione,
rilasciando l'applicazione in produzione.

\section{Release}
Dopo che l'applicazione è stata stabilizzata è possibile procedere con la distribuzione.
In questa fase l'applicazione viene prima firmata digitalmente utilizzando un certificato protetto da chiave privata e poi pubblicata sullo specifico marketplace della piattaforma target. 
La firma dell'applicazione permette ai dispositivi, 
ai mezzi di distribuzione e ai marketplace di sapere quali applicazioni hanno origine dal proprietario di uno specifico certificato e di verificare che il codice non sia stato modificato da quando è stato firmato~\cite{mednieks2011programming}.

\section{Monitoraggio}
La fase di monitoraggio (e manutenzione) è quella più lunga e dispendiosa in termini di tempo e risorse. 
Per le applicazioni mobile esistono alcune situazioni che rendono il monitoraggio più complesso rispetto ad altre tipologie di applicazioni, 
come ad esempio le Web Application. 
Bisogna infatti considerare che\footnote{\href{https://www.datadoghq.com/blog/mobile-monitoring-best-practices/}{https://www.datadoghq.com/blog/mobile-monitoring-best-practices/}}:

\begin{itemize}
    \item le applicazioni mobile eseguono su una vasta gamma di dispositivi con caratteristiche diverse e può essere quindi difficile ottenere una chiara visibilità delle prestazioni lato client,
    
    \item se gli utenti riscontrano un problema, mentre una patch per applicazioni web può essere distribuita quasi all'istante, la distribuzione degli aggiornamenti delle applicazioni mobile richiede tempo e soprattutto l'attivazione da parte degli utenti per scaricarli.
\end{itemize}

Per effettuare un monitoraggio efficace è necessario misurare e controllare continuamente le performance dell'applicazione mobile, 
il comportamento dell'utente e gli errori che essi riscontrano. 
Alcuni esempi di metriche fondamentali sono: 
tempo d'avvio dell'applicazione, 
network performance, 
utilizzo delle risorse (CPU, disco e memoria) e metriche custom derivanti dalle azioni dell'utente.