% !TeX root = ../main.tex

L'output finale di questa tesi dimostra che è stato possibile adottare le pratiche e gli strumenti abilitanti la cultura DevOps per il processo di sviluppo di applicazioni mobile multipiattaforma. Il sistema di automazione realizzato è stato in grado infatti di compilare, testare, impacchettare e rilasciare una applicazione, sia in versione Android che in versione iOS, mediamente in circa 28 minuti.

Per poter raggiungere questo risultato decisamente positivo è stata necessaria una prima fase di analisi approfondita sulla cultura DevOps, sul ciclo di sviluppo delle applicazioni mobile e sui framework multipiattaforma. Successivamente è stato individuato un caso di studio industriale per i quali sono stati raccolti i requisiti e i vincoli aziendali sia per il processo di sviluppo che per l'applicazione da realizzare. In questa fase sono stati definiti gli obiettivi reali del caso di studio adeguando le necessità dell'azienda alle pratiche e tecniche descritte nella fase iniziale di studio. L'infrastruttura aziendale a supporto dei già esistenti sistemi di automazione adottati in altri ambiti di sviluppo software ha fornito un solido punto di partenza per la realizzazione del caso di studio. In questa infrastruttura sono stati integrati i componenti in grado di soddisfare tutte le esigenze e i vincoli delle applicazioni mobile al fine di automatizzare il processo di sviluppo in ambiente macOS.
Da questo punto di vista,
la soluzione realizzata ha dimostrato la possibilità di realizzare un sistema di automazione in ambiente macOS,
ma ha evidenziato anche che la complessità della configurazione
e della gestione di questi componenti
rendono il modello as-a-Service preferibile,
nonostante i costi elevati rispetto a quello self-hosted.

Il paradigma multipiattaforma scelto per lo sviluppo dell'applicazione mobile
ha contribuito al raggiungimento del risultato finale
grazie alla condivisione della logica applicativa fra le diverse piattaforme.
%
Il bisogno di dover scrivere codice extra
per poter riutilizzare correttamente il modulo condiviso nella applicazione iOS
ha comunque evidenziato alcuni dei limiti di gioventù del framework utilizzato.
%
La vera sfida affrontata nell'automatizzare il processo di sviluppo di applicazioni mobile
è stata la gestione e la configurazione dei numerosi strumenti utilizzati nelle varie fasi del processo.
%
Le due piattaforme infatti,
richiedono strumenti diversi adibiti alle medesime funzioni,
fattore che aumenta notevolmente la complessità del sistema di automazione.
%
Come nel caso dei framework multipiattaforma sono nati anche dei tool,
come quello utilizzato in questa tesi,
con lo scopo di diminuire la complessità del processo di sviluppo da un unico punto che funge da involucro per tutti gli strumenti richiesti dalle diverse piattaforme.