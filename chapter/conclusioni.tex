% !TeX root = ../main.tex

L'output finale di questa tesi dimostra che è stato possibile adottare le pratiche e gli strumenti abilitanti la cultura DevOps per il processo di sviluppo di applicazioni mobile multipiattaforma. Per poter ottenere un sistema di automazione in grado di compilare, testare, impacchettare e rilasciare una applicazione, sia in versione Android che in versione iOS, in circa 28 minuti in media come descritto nei risultati raggiunti (capitolo \ref{ch:risultati}) è stata necessaria una prima fase di analisi approfondita sulla cultura DevOps, sul ciclo di sviluppo delle applicazioni mobile e sui framework multipiattaforma. 

Successivamente è stato individuato un caso di studio per dimostrare l'efficacia della cultura DevOps applicata allo sviluppo mobile multipiattaforma in un contesto industriale. In questa fase sono stati raccolti i requisiti e i vincoli aziendali sia per il processo di sviluppo che per l'applicazione da realizzare in modo da adeguarli a quanto appreso dalla prima fase di studio e definire gli obiettivi reali.




%Per poter misurare con precisione i vantaggi derivanti dalla loro adozione, il sistema è stato realizzato appositamente in modo tale da poter essere integrato velocemente nel processo di sviluppo di altri team in azienda con lo scopo di raccogliere metriche da confrontare con quelle descritte nel capitolo precedente. 