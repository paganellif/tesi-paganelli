% !TeX root = ../main.tex

\section{Introduzione}
% cos'è devops, la filosofia devops, no silos, ecc
A causa della continua evoluzione delle tecnologie e all'aumento della concorrenza le aziende hanno bisogno di realizzare prodotti sempre più velocemente mantenendo e possibilmente aumentando la qualità~\cite{krief2019learning}. In risposta a questo problema è nata una vera e propria cultura, chiamata DevOps, la quale definisce sia un modo di pensare che una metodologia di lavoro fortemente basata sulla collaborazione tra i diversi team. \\
Fornisce infatti un insieme di pratiche al fine di ridurre le barriere tra il team di sviluppo (Dev) e il team operativo (Ops). Da un lato gli sviluppatori vogliono innovare e rilasciare software velocemente, mentre dall'altro lato l'obiettivo è quello di garantire la stabilità e la qualità dei sistemi.\\
Oltre alla collaborazione tra i diversi team esistono altri principi alla base della cultura DevOps:
\begin{itemize}
    \item \textbf{Automazione} - L'intervento umano deve essere ridotto il più possibile al fine di ridurre gli errori e le tempistiche.
    \item \textbf{Misurazione} - Deve essere possibile analizzare in ogni momento lo stato dei processi che compongono il sistema.
    \item \textbf{Monitoraggio} - 
\end{itemize}

\section{Tecniche di automazione e vantaggi}
Ogni azienda ha i propri vincoli, requisiti e metodi i quali la rendono unica e per questo motivo non esiste un unico modo di "fare DevOps". Esistono invece diverse tecniche che rispettano i principi DevOps e che possono essere applicate e adattate ad ogni azienda al fine di aumentare la qualità del software prodotto e diminuire il time-to-market (TTM)~\cite{devis2016effective}.

\subsection{Continuous Integration}

\subsection{Continuous Delivery}

\subsection{Continuous Inspection}

\section{Strumenti}
% descrizione strumenti per l'automazione in generale (gitlab runner + Pipeline as Code)
\subsection{Pipeline as Code}

\subsection{Runners}

\subsection{Software as a Service}
% descrizione strumenti as a service per l'automazione del processo (runner managed, sonarcloud, ...) e dello dello sviluppo mobile (xcode cloud, bitrise, google play console, testflight)
