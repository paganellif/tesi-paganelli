\newpage
\chapter{Capitolo}

\section{Analisi dei Requisiti}
\begin{description}
    \item[R1] - Visualizzare documenti.
    \begin{description}
        \item[R1.1] - In modo fluido, mostrando il contenuto adattato al dispositivo in cui viene mostrato.
        \item[R1.2] - In modo statico, mostrando il contenuto con uno specifico layout indipendente dal dispositivo in cui viene mostrato.
    \end{description}
    \item[R2] - Modificare documenti in modo fluido.
    \begin{description}
        \item[R2.1] - Aggiungere commenti, sottolineature, evidenziazioni, al contenuto fluido.
        \item[R2.2] - Memorizzare commenti, sottolineature, evidenziazioni apportate al contenuto fluido.
    \end{description}
    \item[R3] - Gestione utente.
    \begin{description}
        \item[R3.1] - Login (autenticazione) utente.
        \item[R3.2] - Visualizzare documenti a cui l'utente è abbonato.
    \end{description}
    \item[R4] - Ricerca documenti
    \begin{description}
        \item[R4.1] - Ricerca documenti senza autenticazione. Qualsiasi utente può effettuare una ricerca dei documenti disponibili, senza ottenere il contenuto.
        \item[R4.2] - Ricerca avanzata in base a diversi campi %TODO: da definire quali campi
    \end{description}
    % Inserire anche i requisiti della conversione da documento statico (pdf/html) a fluido (epub)?
    % Lo sviluppo di questo microservizio fa parte del lavoro di progetto/tesi?
\end{description}


\begin{description}
    \item[T1] - 
\end{description}