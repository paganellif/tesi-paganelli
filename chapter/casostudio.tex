% !TeX root = ../main.tex

\section{Introduzione}
% situazione attuale (esistono team di sviluppo mobile in maggioli che non hanno automazione e non usano tecniche multiplatform) e cosa si vuole fare (si vuole fornire un modello di processo automatizzato e un esempio di utilizzo di strumenti multiplatform)

In questo capitolo vengono inizialmente descritte le motivazioni che hanno spinto l'azienda Maggioli S.p.A.\footnote{\href{https://www.maggioli.com/en-us}{https://www.maggioli.com/en-us}} a dedicare risorse per la ricerca e la sperimentazione nei campi delle applicazioni multipiattaforma e delle tecniche DevOps in ambito mobile. Successivamente sono indicati i requisiti del caso di studio industriale individuato, il quale può essere suddiviso in due macroaree:
\begin{itemize}
    \item definizione del processo di sviluppo per applicazioni multipiattaforma tramite l'adozione della cultura DevOps,
    \item sviluppo di una applicazione mobile multipiattaforma utilizzando il processo di sviluppo definito.
\end{itemize}

\section{Contesto aziendale}
% descrizione del contesto maggioli editore, editoria digitale e lo scopo della app
Tra i core business dell'azienda Maggioli S.p.A. è rimasto centrale il ruolo dell'editoria ma col trascorrere degli anni e il mutare delle esigenze dei clienti, i quali sono principalmente pubblica amministrazione (PA) e professionisti privati, come avvocati, architetti, commercialisti ed ingegneri edili, si è verificata una transizione verso il mondo digitale.

I servizi digitali erogati per la consultazione delle pubblicazioni hanno superato considerevolmente il formato cartaceo il quale rimane comunque un metodo secondario di consultazione disponibile seppur in forma molto ridotta. Per l'editoria digitale esiste in Maggioli una business unit dedicata chiamata \textit{Digital Media} il cui ruolo principale consiste nella realizzazione e manutenzione dei siti Web Maggioli dedicati alla ricerca e alla visualizzazione delle pubblicazioni digitali. I seguenti sono soltanto alcuni dei siti gestiti dal team \textit{Digital Media}: Biblioteca Digitale\footnote{\href{https://bibliotecadigitale.maggioli.it/}{https://bibliotecadigitale.maggioli.it/}}, Appalti \& Contratti\footnote{\href{https://www.appaltiecontratti.it/}{https://www.appaltiecontratti.it/}} e Periodici\footnote{\href{https://www.periodicimaggioli.it/}{https://www.periodicimaggioli.it/}}. 

\section{Requisiti processo di sviluppo}
% requisiti processo di sviluppo: automazione build, test (unit+ui), delivery sui vari store, analisi sonarqube, ...

%\section{Configurazione Infrastruttura}
%\subsection{Self-Hosted MacOS GitLab Runner}
% runner macos: in azienda non esistono runner macos.. riprendere il problema del fatto che apple obbliga a usare macOS, indicare le possibili soluzioni (runner managed/self-hosted, ecc) e come ho configurato il runner self-hosted
% approfondire tipologie di runner executor in gitlab e perche ho usato l'executor shell

%\subsection{SonarQube}
% sonarqube aziendale per l'analisi del codice

\section{Requisiti applicazione multipiattaforma}
% requisiti della applicazione da sviluppare, come sono stati identificati, ... (reader, epub, segnalibri, ecc)