% !TeX root = ../main.tex

\section{Introduzione}
% situazione attuale (esistono team di sviluppo mobile in maggioli che non hanno automazione e non usano tecniche multiplatform) e cosa si vuole fare (si vuole fornire un modello di processo automatizzato e un esempio di utilizzo di strumenti multiplatform)

\section{Contesto aziendale}
% descrizione del contesto maggioli editore, editoria digitale e lo scopo della app

\section{Requisiti processo di sviluppo}
% requisiti processo di sviluppo: automazione build, test (unit+ui), delivery sui vari store, analisi sonarqube, ...

%\section{Configurazione Infrastruttura}
%\subsection{Self-Hosted MacOS GitLab Runner}
% runner macos: in azienda non esistono runner macos.. riprendere il problema del fatto che apple obbliga a usare macOS, indicare le possibili soluzioni (runner managed/self-hosted, ecc) e come ho configurato il runner self-hosted
% approfondire tipologie di runner executor in gitlab e perche ho usato l'executor shell

%\subsection{SonarQube}
% sonarqube aziendale per l'analisi del codice

\section{Requisiti applicazione}
% requisiti della applicazione da sviluppare, come sono stati identificati, ... (reader, epub, segnalibri, ecc)