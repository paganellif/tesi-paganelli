% !TeX root = ../main.tex

In questo capitolo viene analizzato il processo di sviluppo tipico per le applicazioni mobile al fine di porre le basi per la progettazione di un flusso di sviluppo il più automatizzato possibile nel capitolo successivo. Lo scopo di questa fase è quindi la definizione di tutti i task necessari e di tutti gli strumenti utilizzati nello sviluppo di applicazioni mobile, dalla scrittura del codice sorgente al rilascio sui marketplace delle relative piattaforme target (Android e iOS).
% quali parti del processo sono in comune? quali tool servono? 
% parlare qui del domain driven design o parlarne nel capitolo 5 dicendo che è stato adottato per il poc?
\section{Analisi e Pianificazione}
\section{Progettazione UI/UX}
\subsection{Prototipazione}
% mockup e validazione mockup da parte degli utenti/tester
\section{Sviluppo}
\subsection{Ambiente di sviluppo}
% descrivere setup ide e tools (android studio,java sdk, kotlin sdk, gradle, plugin vari, xcode, emulatori vari, ...)
\section{Testing}
\subsection{Unit Testing}
\subsection{End-to-End/UI Testing}
\section{Rilascio}
% signing app, chiavi, store, beta, ...
\subsection{Alpha Release}
\subsection{Beta Release}
\subsection{Production Release}

\section{Monitoraggio/Analytics}
% monitoraggio sia della applicazione che dell'utente (analytics)