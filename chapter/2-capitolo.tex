% !TeX root = ../main.tex

In questo capitolo viene analizzato il processo di sviluppo tipico per le applicazioni mobile al fine di porre le basi per la progettazione di un flusso di sviluppo il più automatizzato possibile nel capitolo successivo. Lo scopo di questa fase è quindi la definizione di tutti i task necessari e di tutti gli strumenti utilizzati nello sviluppo di applicazioni mobile, dalla scrittura del codice sorgente al rilascio sui marketplace delle relative piattaforme target (Android e iOS).
% quali parti del processo sono in comune? quali tool servono? 
% parlare qui del domain driven design o parlarne nel capitolo 5 dicendo che è stato adottato per il poc?
\section{Analisi e Pianificazione}
\section{Progettazione UI/UX}
\subsection{Prototipazione}
% mockup e validazione mockup da parte degli utenti/tester
\section{Sviluppo}
\subsection{Ambiente di sviluppo}
% descrivere setup ide e tools (android studio,java sdk, kotlin sdk, gradle, plugin vari, xcode, emulatori vari, ...)
\section{Testing}
\subsection{Unit Testing}
\subsection{UI Testing}
\section{Rilascio}
% signing app, chiavi, store, beta, ...
\subsection{Alpha Release}
La prima versione funzionante di una applicazione è detta alpha ed è utilizzata per il testing interno di specifiche funzionalità. Questo significa che può presentare anche bug o funzionalità mancanti ma almeno deve contenere le funzionalità che devono essere testate per quella specifica versione alpha. \\
Solitamente prima della release di una applicazione, anche se in fase di testing, è necessario attendere la sua approvazione da parte del gestore del servizio. Il processo di approvazione pre-release è detto \textit{App Review}. Nel caso del testing interno è possibile distribuire la applicazione ad un insieme ristretto di tester.\\
Continuando con i rilasci di versioni alpha vengono aggiunte nuove funzionalità e/o risolti eventuali bug: quando la versione è considerata pronta viene eseguita una sua promozione. Con promozione si intende in questo caso il rilascio di una versione alpha in versione beta.

\subsection{Beta Release}
A questo punto del processo di sviluppo la applicazione è considerata completa a tutti gli effetti a meno di bug e/o problemi di stabilità. La versione beta rappresenta dunque la prima versione della applicazione resa disponibile ai tester esterni, ovvero quegli utenti che non hanno partecipato alle fasi di sviluppo e che svolgono il ruolo di validazione delle funzionalità.\\
Si distinguono due tipologie di beta testing:
\begin{itemize}
    \item \textit{Aperto} - La applicazione è rilasciata per la fase di testing esterno permettendo l'accesso a qualsiasi utente con account da beta tester. Nel caso di Android per poter testare una applicazione in versione beta aperta è necessario disporre di un account Google Developer.
    \item \textit{Chiuso} - L'accesso alla applicazione di test è limitato ad un insieme ristretto di tester, tipicamente gestiti tramite mailing list o link di condivisione.
\end{itemize}
Dopo aver ottenuto la validazione da parte dei tester, la quale potrebbe richiedere più iterazioni di sviluppo e rilascio di versioni alpha-beta, anche per la versione beta si effettua la promozione, rilasciando la applicazione in produzione.

\subsection{Production Release}



\section{Monitoraggio/Analytics}
% monitoraggio sia della applicazione che dell'utente (analytics)